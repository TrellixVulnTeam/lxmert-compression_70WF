% !TeX root=main.tex

\chapter{پیش‌زمینه}
\thispagestyle{empty}

\section{ساز و کار توجه
}\label{attention}
هدف استفاده از ساز و کار توجه
\LTRfootnote{\lr{Attention}}
بازیابی اطلاعات از بردارهای زمینه
\LTRfootnote{\lr{Context Vector}}
\lr{${y_j}$}
در رابطه با بردار پرس‌وجو
\LTRfootnote{\lr{Query Vector}}
\lr{$x$}
می‌باشد. ساز و کار توجه ابتدا امتیاز 
\lr{$\alpha_j$}
را بین بردار پرس‌وجو 
\lr{$x$}
و بردار زمینه
\lr{$y_j$}
محاسبه می‌کند.
\begin{equation}
	a_j = Score(x, y_j)
\end{equation}
\begin{equation}
	\alpha_j = \frac{exp(a_j)}{\sum_{k} exp(a_k)}
\end{equation}
خروجی لایه توجه میانگین وزن‌دار امتیاز 
\lr{$\alpha_j$}
به ازای بردار‌های زمینه می‌باشد. محاسبات انجام شده مشابه لایه 
\lr{softmax}
است.
\begin{equation}\label{self}
	Att_{x \to \{y_j\}} = \sum_{j} \alpha_jy_j
\end{equation}


اگر بردار پرس‌وجو
\lr{$x$}
مجموعه‌ای از بردار زمینه
\lr{$\{y_j\}$}
باشد، امتیاز به دست آمده از معادله
\ref{self}
 توجه به خود\LTRfootnote{\lr{self-attention}}
  نامیده می‌شود.




