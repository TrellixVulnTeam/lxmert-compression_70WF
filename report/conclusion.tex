% !TeX root=main.tex
\chapter{نتیجه‌گیری و پیشنهادات}
\thispagestyle{empty}

\section{نتیجه‌گیری}

با توجه به نتایج به دست‌آمده از آزمایش‌های انجام شده مشاهده می‌شود می‌توان 50 درصد اتصالات کم‌وزن شبکه
\lr{LXMERT}
را حذف کرد. عملکرد شبکه هرس شده در این حالت مشابه 95 درصد شبکه کامل می‌شود. بنابراین برای کاهش پارامتر‌ها و اندازه شبکه حذف اتصالات کم‌وزن یکی از بهترین روش‌های ممکن است. ولی اگر اتصالات به صورت تصادفی اتخاب شود نتایج متفاوت است. بدین معنی که علاوه بر اتصات نوع اتصالاتی که در زیرشبکه هرس شبکه نگه می‌داریم عاملی مهم و تاثیر گذار است. 

\section{پیشنهادات و کار‌های آینده}
با توجه به پیشرفت چشم‌گیر در هوش مصنوعی و حرکت سریع به سمت استفاده از ابزار‌های هوشمند، برای ادامه تحقیقات موارد زیر پیشنهاد می‌شود.
\begin{enumerate}
	\item در این پژوهش مدل
	\lr{LXMERT}
	بر روی مجموعه داده
	\lr{VQA}
	بررسی شد. در ادامه می‌توان دو مجموعه داده دیگر از جمله
	\lr{GQA}
	و 
	\lr{NLVR2}
	را بررسی کرد.
	\item
	می‌توان کارایی شبکه هرس شده و آموزش دیده روی مسئله 
	\lr{VQA}
	را بر روی سایر مجموعه داده‌ها
	\lr{(GQA, NLVR2)}
	 بررسی نمود. بدین ترتیب تاثیر انتقال یادگیری
	\LTRfootnote{\lr{Transfer Learning}}
	 در 
	\lr{LXMERT}
	مشخص می‌شود.
	\item روش بررسی شده در این پژوهش، هرس اتصالات شبکه می‌باشد. از این رو زمان اجرا ابتدا به انتها شبکه تفاوتی چندانی نمی‌کند. می‌توان انواع دیگر هرس از جمله هرس ساختاری
	\LTRfootnote{\lr{Structural Pruning}}
	را مورد بررسی قرار داد.
%		\lr{head attention}
	\item
	نتایج به دست‌آمده در روش اتصالات با وزن زیاد (بخش \ref{high_mag_pruning}) دور از انتظار بود. می‌توان در ادامه تحقیقات معماری شبکه
	\lr{LXMERT}
	را به صورت دقیق‌تر بررسی کرد. این بررسی ممکن است به معرفی مدل دیگری با ساختار جدید و بهبود دقت در مسئله پرسش و پاسخ تصویری ختم شود.
	\item بیشتر پژوهش‌ها در موضوع فشرده‌سازی شبکه به صورت تئوری است. وقت آن است  نتیجه این پژوهش‌ها در عمل و برنامه‌های کاربردی
	\LTRfootnote{\lr{Application}}
	 مورد استفاده در روزمره انسان‌ها مورد بررسی قرار گیرد. 
	
	\item مجموعه داده
	\lr{VQA}
	که در آزمایش‌ها مورد استفاده قرار گرفت به زبان انگلیسی می‌باشد. با توجه به نبود مجموعه داده مناسب به زبان فارسی، یکی دیگر از کار‌های ارزشمند جمع‌آوری مجموعه داده فارسی پرسش و پاسخ تصویری می‌باشد. بدین ترتیب برنامه‌های کاربردی طراحی شده برای کمک به کم‌بینایان، نابینایان یا استفاده‌های دیگر می‌توانند به زبان فارسی باشند.
\end{enumerate}



