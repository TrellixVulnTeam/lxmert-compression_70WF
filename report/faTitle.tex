% !TeX root=main.tex

\university{علم و صنعت ایران}

\faculty{دانشکده مهندسی کامپیوتر}

\department{گروه هوش مصنوعی}

\subject{مهندسی کامپیوتر}

\field{هوش مصنوعی}

\title{بررسی هرس شبکه عصبی در پرسش و پاسخ تصویری}

\firstsupervisor{دکتر سید صالح اعتمادی}

\name{غزاله}

\surname{محمودی}

\studentID{96522249}

\thesisdate{شهریور 1400}

\projectLabel{گزارش پروژه پایانی}

\firstPage
\besmPage
\davaranPage

\vspace{.5cm}

%\renewcommand{\arraystretch}{1.2}
\begin{center}
	\begin{tabular}{| p{8mm} | p{18mm} | p{.17\textwidth} |p{14mm}|p{.2\textwidth}|c|}
		\hline
		ردیف	& سمت & نام و نام خانوادگی & مرتبه \newline دانشگاهی &	دانشگاه یا مؤسسه &	امضـــــــــــــا\\
		\hline
		۱  &	استاد راهنما & دکتر \newline  صالح اعتمادی & استادیار & دانشگاه \newline علم و صنعت ایران &  \\
		\hline

	\end{tabular}
\end{center}


\keywords{پرسش و پاسخ تصویری، شبکه
\lr{LXMERT}، فرضیه بلیت قرعه‌کشی.}
\fa-abstract{
مدل‌های از قبل آموزش دیده در مقیاس بزرگ مانند
\lr{LXMERT}
در حال محبوب شدن برای یادگیری بازنمایی‌های متن و تصویر هستند. این مدل‌ها در مسايل مشترک بین بینایی و زبان کاربرد دارند. بر اساس فرضیه بلیط قرعه کشی
\LTRfootnote{\lr{Lottory Thicket Hypothesis}}،
 شبکه‌های عصبی حاوی زیرشبکه‌های\LTRfootnote{\lr{subnetwork}} 
 کوچکتری هستند که قادرند با آموزش در انزوا
\LTRfootnote{\lr{isolation}}
  عملکردی مشابه شبکه کامل آموزش دیده داشته باشند. در این پروژه، وجود چنین زیرشبکه‌ای در شبکه
   \lr{LXMERT}
   که بر روی مسئله پرسش و پاسخ تصویری آموزش دیده، بررسی می‌شود. همچنین مقادیر مختلف هرس شبکه و تاثیر آن بر کارایی شبکه مورد ارزیابی قرار می‌گیرد.
}

\abstractPage

\newpage\clearpage

